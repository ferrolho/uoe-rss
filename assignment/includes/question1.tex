\section{Forward and inverse kinematics}

\subsection{(10 marks)}

\begin{center}
    $ q =
    \begin{pmatrix}
         0.66 \\
        -0.44 \\
         1.06 \\
         1.33
    \end{pmatrix} $
\end{center}

\begin{align*}
    T_{base \rightarrow L_1} &=
        \begin{bmatrix}
            cos(q_1) & - sin(q_1) & 0 & 0 \\
            sin(q_1) &   cos(q_1) & 0 & 0 \\
                   0 &          0 & 1 & 0.16 \\
                   0 &          0 & 0 & 1 \\
        \end{bmatrix} \\
    \\
    T_{L_1 \rightarrow L_2} &=
        \begin{bmatrix}
              cos(q_2) & 0 & sin(q_2) & 0 \\
                     0 & 1 &        0 & 0 \\
            - sin(q_2) & 0 & cos(q_2) & 0.15 \\
                     0 & 0 &        0 & 1 \\
        \end{bmatrix} \\
    \\
    T_{L_2 \rightarrow L_3} &=
        \begin{bmatrix}
              cos(q_3) & 0 & sin(q_3) & 0 \\
                     0 & 1 &        0 & 0 \\
            - sin(q_3) & 0 & cos(q_3) & 0.2 \\
                     0 & 0 &        0 & 1 \\
        \end{bmatrix} \\
    \\
    T_{L_3 \rightarrow L_4} &=
        \begin{bmatrix}
              cos(q_4) & 0 & sin(q_4) & 0 \\
                     0 & 1 &        0 & 0 \\
            - sin(q_4) & 0 & cos(q_4) & 0.15 \\
                     0 & 0 &        0 & 1 \\
        \end{bmatrix} \\
    \\
    T_{L_4 \rightarrow gripper} &=
        \begin{bmatrix}
            1 & 0 & 0 & 0 \\
            0 & 1 & 0 & 0 \\
            0 & 0 & 1 & 0.35 \\
            0 & 0 & 0 & 1
        \end{bmatrix}
\end{align*}

\begin{align*}
    T_{base \rightarrow gripper} &=
        T_{base \rightarrow L_1} \cdot
        T_{L_1 \rightarrow L_2} \cdot
        T_{L_2 \rightarrow L_3} \cdot
        T_{L_3 \rightarrow L_4} \cdot
        T_{L_4 \rightarrow gripper}\\
        &=
        \begin{bmatrix}
            -0.29243998 & -0.61311685 &  0.73387096 & 0.25840905 \\
            -0.22696411 &  0.78999223 &  0.56956086 & 0.20055253 \\
            -0.92895972 &  0          & -0.37018083 & 0.48346881 \\
             0          &  0          &  0          & 1
        \end{bmatrix}
\end{align*}

And therefore:

\begin{center}
    $ \boldsymbol{y_t} =
        \begin{bmatrix}
            0.25840905 \\
            0.20055253 \\
            0.48346881
        \end{bmatrix} $
\end{center}

\clearpage

% - - - - - - - - - - - - - - - - - - - - - - - - - - -

\subsection{(10 marks)}

Kinematic map $ y = \Phi(q) $:

\begin{center}
    $ y =
        \begin{bmatrix}
            x \\
            y \\
            z
        \end{bmatrix}
        =
        \begin{bmatrix}
            x \\
            y \\
            z
        \end{bmatrix} $
\end{center}

\begin{align*}
    J(q) &=
        \begin{bmatrix}
            \frac{\partial x}{\partial q_1} & \frac{\partial x}{\partial q_2} & \frac{\partial x}{\partial q_3} \\[0.5em]
            \frac{\partial y}{\partial q_1} & \frac{\partial y}{\partial q_2} & \frac{\partial y}{\partial q_3} \\[0.5em]
            \frac{\partial z}{\partial q_1} & \frac{\partial z}{\partial q_2} & \frac{\partial z}{\partial q_3} \\[0.5em]
        \end{bmatrix}\\
        %&= 14
\end{align*}

% - - - - - - - - - - - - - - - - - - - - - - - - - - -

\subsection{(10 marks)}

\newpage
